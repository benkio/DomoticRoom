\section{Testing}
\labelsec{Testing}
%===========================================================================

\subsection{Server}

Per quanto riguarda la parte di testing del server si sono implementati e impostati per lo sviluppo alcuni test unitari dell'applicativo. Il codice si pu\`o visionare all'interno dell'apposita cartella di test.

Particolare attenzione si \`e poi posta per i test di integrazione incui si \`e sviluppato un semplice script in fsharp che consente di simulare l'invio di dati dai valori random, strutturati come indicato nella sezione del progetto, all'endpoint specifico del server che si occupa di ricevere e salvare correttamente i dati. Questo ha consentito di parallelizzare al meglio il lavoro tra i membri del gruppo inquanto \`e possibile testare l'applicativo senza dover ottenere dei dati reali e conseguentemente impostare tutta l'infrastruttura ed inoltre ha impostato uno standard da seguire per l'invio dei dati verso il server che i client devono seguire affinch\'e non si verifichino problemi. Anche questo script \`e presente nel repository con le relative dipendenze.
