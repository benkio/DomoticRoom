\section{Problem Analysis}
\labelsec{ProblemAnalysis}
%===========================================================================
\subsection{Logic architecture}
\subsection{Abstraction gap}

In questa sezione aggiungeremo tutti i tipi di astrazioni che sono richieste per affrontare il progetto e che non sono direttamente fruibili attraverso la tecnologia di riferimento.

\begin{enumerate}
  \item Web Server: Vista la necessit\'a di comunicare attraverso la rete \'e necessario che si utilizzi un paradigma a message-passing, soprattutto per la comunicazione che avviene tra il raspberry e il server. Chiaramente vogliamo cercare di sfruttare qualche tecnologia gi\'a esistente che ci garantisca di elevarci rispetto ad un livello basso come pu\'o essere l'utilizzo di socket.
  \item Continuous Integration, Testing and collaborative source control: Lavorando in gruppo sullo stesso repository \'e necessario impostare il lavoro affinch\'e sia possibile effettuare le modifiche in maniera indipendente gli uni dagli altri e allo stesso modo sia possibile controllare automaticamente che i test predisposti e le modifiche effettuate siano coerenti con le specifiche e che il building del progetto sia in ogni caso garantito.
\end{enumerate}

\subsection{Risk analysis}

In questa sezione vanno elencati invece i rischi che si possono intraprendere in un progetto di questo tipo e quindi formalizzarli fin da subito, prima di eseguire l'effettiva realizzazione dello stesso in modo da poterli affontare e discutere preventivamente per essere pronti se questi si verificano.

\begin{itemize}
  \item Aumento dei Costi: \'e necessario porre particolare attenzione alla struttura hardware del sistema e di come i singoli componenti vanno ad interconnettersi assieme per evitare che si debbano affrontare dei costi aggiuntivi, una volta che il progetto \'e gi\'a avviato, causati da una qualche mancanza o per via di un'estensione del progetto in corso d'opera.
\end{itemize}

