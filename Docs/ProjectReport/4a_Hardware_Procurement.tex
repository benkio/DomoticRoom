\documentclass[10pt]{article}
\usepackage[usenames]{color} %usato per il colore
\usepackage{amssymb} %maths
\usepackage{amsmath} %maths
\usepackage[utf8]{inputenc} %utile per scrivere direttamente in caratteri accentuati
\begin{document}
\section{Acquisto Hardware}
\labelsec{Hardware Procurement}

Sfortunatamente il primo probelma incontrato in un progetto come il seguente e' stata la necessit\'a di acquistare la parte hardware del sistema che si andrà a costruire. Di consequenza si e' messo in atto un processo di ricerca dei sensori, cavi e quant'altro per riuscire a soddisfare i requisiti

\subsection{Dispositivi di Computazione}

Prima di tutto necessitiamo di un dispositivo in grado di computare i dati emessi dai vari sensori e che sia interamente programmabile. Nel corso abbiamo visto due possibilita' che hanno avuto molto successo recentemente:

\begin{itemize}
  \item Arduino
  \item Raspberry Pi
\end{itemize}

Abbiamo scelto la seconda opzione data la maggior familiarita' con il dispositivo e dal momento in cui risulta piu' facile il riutilizzo dello stesso una volta terminato questo progetto.

Costo del dispositivo: 44,50 \euro

\subsection{Sensori}

Un'altra cosa fondamentale riguarda i sensori necessari per catturare i parametri richiesti. Abbiamo Quindi scelto i sequenti sensori

\begin{table}[]
\centering
\begin{tabular}{lll}
\cline{1-1}
\multicolumn{1}{|c|}{\textbf{Parametri Ambientali}} & \multicolumn{1}{c}{\textbf{Sensori}} & \multicolumn{1}{c}{\textbf{Costo}} \\ \cline{1-1}
Temperatura                               &                                     &                                   \\
Luce                                      &                                     &                                   \\
Movimento                                 &                                     &                                
\end{tabular}
\caption{Sensor Table}
\label{Sensor Table}
\end{table}

\subsection{Hardware Aggiuntivo}

\begin{table}[]
\centering
\begin{tabular}{lll}
\multicolumn{1}{c}{\textbf{Hardware}} &  & \multicolumn{1}{c}{\textbf{Costo}} \\
Breadboard                            &  &                                   \\
Wires                                 &  &                                   \\
Resistors                             &  &                                  
\end{tabular}
\caption{Addictional Hardware}
\label{Addictional Hardware}
\end{table}

\newpage

\end{document}